\documentclass[]{article}

% Imported Packages
%------------------------------------------------------------------------------
\usepackage{amssymb}
\usepackage{amstext}
\usepackage{amsthm}
\usepackage{amsmath}
\usepackage{enumerate}
\usepackage{fancyhdr}
\usepackage[margin=1in]{geometry}
\usepackage{graphicx}
\usepackage{extarrows}
\usepackage{setspace}
%------------------------------------------------------------------------------

% Header and Footer
%------------------------------------------------------------------------------
\pagestyle{plain}  
\renewcommand\headrulewidth{0.4pt}                                      
\renewcommand\footrulewidth{0.4pt}                                    
%------------------------------------------------------------------------------

% Title Details
%------------------------------------------------------------------------------
\title{Deliverable \#1 Template}
\author{SE 3A04: Software Design II -- Large System Design}
\date{}                               
%------------------------------------------------------------------------------

% Document
%------------------------------------------------------------------------------
\begin{document}

\maketitle	

\section{Introduction}
\label{sec:introduction}
% Begin Section

The following is a description of the product to be developed, as well as an overview of the SRS.

\subsection{Purpose}
\label{sub:purpose}
	[[spacebase13]] is a mobile simulation game that models the overall behavior of a space settlement on a celestial body, i.e. a real-life system. The game represents how different sub-systems interact with each other and affect the overall system. It replicates how each of the sub-systems of a space base react to different stimuli from outside of the system. Therefore, the player must ensure that the sub-systems are working in the desired fashion to keep the base operating. The purpose of this document is to provide a general description of this project and to specify the requirements for this game. It is meant to be a form of communication between the developers of this game and its clients, i.e. Dr. Ridha Khedri and the teaching assistants for SFWR ENG 3A04. The SRS is also meant to be used as a reference by the developers to ensure that the specified requirements have been fulfilled.

\subsection{Scope}
\label{sub:scope}
	The focus of this SRS is the development of the software product, [[spacebase13]. The game mimics the operations of a space settlement, and places the user as the head of the base. Hence, the key duty of the player is to ensure that the base remains operating. The player does this by keeping an eye on all the sub-systems and maintaining them whenever there is a need. This will be done by assigning tasks to all the members of the base, which could be anything from fixing a breach in the walls of the base to interaction with alien flora and fauna. If there are multiple stimuli happening concurrently, then the player must prioritize the tasks, based on how critical they are and on what level they affect the safety of the community. The main objective of this game is to provide entertainment to its users. However, it will also develop time management skills of the users, as well as improve their multitasking abilities. 

\subsection{Definitions, Acronyms, and Abbreviations}
\label{sub:definitions_acronyms_and_abbreviations}
	None were used for this document.

\subsection{References}
\label{sub:references}
	No references were used for this document.

\subsection{Overview}
\label{sub:overview}
	This document describes the product software that is to be developed as well as the requirements specifications.
\begin{itemize}
	\item The second section of this document provides an overall description of the game. It provides product perspective when compared to another related product. It describes major functions of the product as well as the user characteristics. It also explains the constraints of the system as well as the assumptions for this project.
	\item The third section of this document specifies the functional software requirements of this game. It provides sufficient details to design a system with the specific requirements and to test that the design has fulfilled those requirements. 
	\item The final section of the document is the non-functional requirements, which vary from different perceptions and are used to ensure successful integration of the game into society.
	\item A Division of Labour section is also placed at the end of the document which details the contributions of each team member.
\end{itemize}

% End Section

\section{Overall Description}
\label{sec:overall_description}

The following is a general description of the product and its requirements. For more specific requirements, see the requirements sections.

\subsection{Product Perspective}
\label{sub:product_perspective}
	[[spacebase13]] is an independent and totally self-contained system. It does not require network communication to function normally. [[spacebase13]] will contain elements of various existing simulation games such as [[dwarf fortress]], but is not intended to completely emulate any of them. The simulation is intended for entertainment purposes only, so bears only a thematic relationship to some scientific simulation software. As [[spacebase13]] is an Android app, it may be distributed on the Google Play Store, but that does not constitute part of the system.

\subsection{Product Functions}
\label{sub:product_functions}
	\begin{enumerate}
		\item The user will be able to view compositions of various sub-views of the system. Each sub-view corresponds to a different subsystem.
		\item The user will be able to stimulate the system. Each stimuli will be able to one or many subsystems, with reactions cascading appropriately. Each subsystem will have at least 1 stimuli.
		\item The application will simulate the various subsystems and their interactions.
		\item Major ways the user will be able to interact with the system include: expanding their station, issuing orders to the population, and managing power and atmospheric controls of their station.
	\end{enumerate}

\subsection{User Characteristics}
\label{sub:user_characteristics}
	Users are expected to have at least a high school level diploma and reading level. Users are expected to be generally familiar with the Android operating system and Android apps. Users are expected to have only a cursory understanding of the subject matter, as [[spacebase13]] is intended for entertainment rather than scientific simulation.

\subsection{Constraints}
\label{sub:constraints}
	The following are constraints on the development of the system.
	\begin{enumerate}
		\item The system must be produced as an Android app.
		\item The system must consist of several separate subsystems. (at least 3)
	\end{enumerate}

\subsection{Assumptions and Dependencies}
\label{sub:assumptions_and_dependencies}
	The following are assumptions that affect the requirements for the system.
	\begin{enumerate}
	\item It is assumed that the device running [[spacebase13]] will have the Android operating system available.
	\item It is assumed that the application will be run with sufficient privileges to read and write necessary files on the device.
	\item  It is assumed that the device will have access to the Google Play Store (or an alternate distribution method if one is chosen).
	\end{enumerate}

\subsection{Apportioning of Requirements}
\label{sub:apportioning_of_requirements}
	The following requirements may be delayed until future versions of the system.
	\begin{enumerate}
		\item Functionality allowing the user to showcase their system (for example on social media).
		\item Functionality allowing the user to swap between multiple saved states. (or to save their state at all)
		\item Non-token graphical features.
		\item More than the minimum number of subsystems.
		\item Ability to enable or disable subsystems at runtime.
	\end{enumerate}

\section{Functional Requirements}
\label{sec:functional_requirements}
% Begin Section

\begin{enumerate}
	\item The user launches the application
	\begin{enumerate}
		\item Android Device OS
			\begin{enumerate}
				\item The system launcher starts the application
			\end{enumerate}
		\item User
			\begin{enumerate}
				\item The system provides a view of all the sub--systems in their present state to the user
				\item The system should allow the user to select a sub-system to interact with
			\end{enumerate}
	\end{enumerate}
	\item The user wants to interact with a sub-system
	\begin{enumerate}
		\item User
			\begin{enumerate}
				\item The system should show all stimuli that can stimulate the sub--system
				\item The system should allow the user to select a stimuli
				\item The system must allow the user to control the stimuli
				\item The system must respond to the stimuli
			\end{enumerate}
	\end{enumerate}
	\item The application is updated
	\begin{enumerate}
		\item Android Device OS
			\begin{enumerate}
				\item The system must notify the operating system that an update is required for the system
			\end{enumerate}
		\item User
			\begin{enumerate}
				\item The system must prompt the user to update the system when the application is launched
			\end{enumerate}
	\end{enumerate}
	\item Time passes within the system
	\begin{enumerate}
		\item User
			\begin{enumerate}
				\item The system should update the attributes that have changes since the last time period
				\item The system should show the user the result of the stimuli from the previous time period
			\end{enumerate}
	\end{enumerate}
	\item The user wants to change settings of the system
	\begin{enumerate}
		\item User
			\begin{enumerate}
				\item The system should provide an interface for editing values of how the system works based on customizable aspects of the system
				\item The system should store these new settings and incorporate them into the system’s functionality
			\end{enumerate}
	\end{enumerate}
	\item The user wants to share snapshot of the system
	\begin{enumerate}
		\item Android Device OS
			\begin{enumerate}
				\item The system should be able to use the OS to interact with social media applications to send data from system to external applications
			\end{enumerate}
		\item Social Media Application
			\begin{enumerate}
				\item The system should be able to provide data that is compatible with the specified social media framework
			\end{enumerate}
			\item User
			\begin{enumerate}
				\item The system should provide a means of sharing user’s system data via social media
			\end{enumerate}
	\end{enumerate}
	\item The user wants to shut down the system
	\begin{enumerate}
		\item Android Device OS
			\begin{enumerate}
				\item The operating system should close the application
			\end{enumerate}
		\item User
			\begin{enumerate}
				\item The system should provide a way to close the current running of the system’s processes
				\item The system should provide a way to save progress within the system and sub--systems
			\end{enumerate}
	\end{enumerate}
\end{enumerate}

% End Section

\section{Non-Functional Requirements}
\label{sec:non-functional_requirements}
% Begin Section
\subsection{Look and Feel Requirements}
\label{sub:look_and_feel_requirements}
% Begin SubSection

\subsubsection{Appearance Requirements}
\label{ssub:appearance_requirements}
% Begin SubSubSection
\begin{enumerate}[{LF}1. ]
	\item 
\end{enumerate}
% End SubSubSection

\subsubsection{Style Requirements}
\label{ssub:style_requirements}
% Begin SubSubSection
\begin{enumerate}[{LF}1. ]
	\item 
\end{enumerate}
% End SubSubSection

% End SubSection

\subsection{Usability and Humanity Requirements}
\label{sub:usability_and_humanity_requirements}
% Begin SubSection

\subsubsection{Ease of Use Requirements}
\label{ssub:ease_of_use_requirements}
% Begin SubSubSection
\begin{enumerate}[{UH}1. ]
	\item 
\end{enumerate}
% End SubSubSection

\subsubsection{Personalization and Internationalization Requirements}
\label{ssub:personalization_and_internationalization_requirements}
% Begin SubSubSection
\begin{enumerate}[{UH}1. ]
	\item 
\end{enumerate}
% End SubSubSection

\subsubsection{Learning Requirements}
\label{ssub:learning_requirements}
% Begin SubSubSection
\begin{enumerate}[{UH}1. ]
	\item 
\end{enumerate}
% End SubSubSection

\subsubsection{Understandability and Politeness Requirements}
\label{ssub:understandability_and_politeness_requirements}
% Begin SubSubSection
\begin{enumerate}[{UH}1. ]
	\item 
\end{enumerate}
% End SubSubSection

\subsubsection{Accessibility Requirements}
\label{ssub:accessibility_requirements}
% Begin SubSubSection
\begin{enumerate}[{UH}1. ]
	\item 
\end{enumerate}
% End SubSubSection

% End SubSection

\subsection{Performance Requirements}
\label{sub:performance_requirements}
% Begin SubSection

\subsubsection{Speed and Latency Requirements}
\label{ssub:speed_and_latency_requirements}
% Begin SubSubSection
\begin{enumerate}[{PR}1. ]
	\item 
\end{enumerate}
% End SubSubSection

\subsubsection{Safety-Critical Requirements}
\label{ssub:safety_critical_requirements}
% Begin SubSubSection
\begin{enumerate}[{PR}1. ]
	\item 
\end{enumerate}
% End SubSubSection

\subsubsection{Precision or Accuracy Requirements}
\label{ssub:precision_or_accuracy_requirements}
% Begin SubSubSection
\begin{enumerate}[{PR}1. ]
	\item 
\end{enumerate}
% End SubSubSection

\subsubsection{Reliability and Availability Requirements}
\label{ssub:reliability_and_availability_requirements}
% Begin SubSubSection
\begin{enumerate}[{PR}1. ]
	\item 
\end{enumerate}
% End SubSubSection

\subsubsection{Robustness or Fault-Tolerance Requirements}
\label{ssub:robustness_or_fault_tolerance_requirements}
% Begin SubSubSection
\begin{enumerate}[{PR}1. ]
	\item 
\end{enumerate}
% End SubSubSection

\subsubsection{Capacity Requirements}
\label{ssub:capacity_requirements}
% Begin SubSubSection
\begin{enumerate}[{PR}1. ]
	\item 
\end{enumerate}
% End SubSubSection

\subsubsection{Scalability or Extensibility Requirements}
\label{ssub:scalability_or_extensibility_requirements}
% Begin SubSubSection
\begin{enumerate}[{PR}1. ]
	\item 
\end{enumerate}
% End SubSubSection

\subsubsection{Longevity Requirements}
\label{ssub:longevity_requirements}
% Begin SubSubSection
\begin{enumerate}[{PR}1. ]
	\item 
\end{enumerate}
% End SubSubSection

% End SubSection

\subsection{Operational and Environmental Requirements}
\label{sub:operational_and_environmental_requirements}
% Begin SubSection

\subsubsection{Expected Physical Environment}
\label{ssub:expected_physical_environment}
% Begin SubSubSection
\begin{enumerate}[{OE}1. ]
	\item 
\end{enumerate}
% End SubSubSection

\subsubsection{Requirements for Interfacing with Adjacent Systems}
\label{ssub:requirements_for_interfacing_with_adjacent_systems}
% Begin SubSubSection
\begin{enumerate}[{OE}1. ]
	\item 
\end{enumerate}
% End SubSubSection

\subsubsection{Productization Requirements}
\label{ssub:productization_requirements}
% Begin SubSubSection
\begin{enumerate}[{OE}1. ]
	\item 
\end{enumerate}
% End SubSubSection

\subsubsection{Release Requirements}
\label{ssub:release_requirements}
% Begin SubSubSection
\begin{enumerate}[{OE}1. ]
	\item 
\end{enumerate}
% End SubSubSection

% End SubSection

\subsection{Maintainability and Support Requirements}
\label{sub:maintainability_and_support_requirements}
% Begin SubSection

\subsubsection{Maintenance Requirements}
\label{ssub:maintenance_requirements}
% Begin SubSubSection
\begin{enumerate}[{MS}1. ]
	\item 
\end{enumerate}
% End SubSubSection

\subsubsection{Supportability Requirements}
\label{ssub:supportability_requirements}
% Begin SubSubSection
\begin{enumerate}[{MS}1. ]
	\item 
\end{enumerate}
% End SubSubSection

\subsubsection{Adaptability Requirements}
\label{ssub:adaptability_requirements}
% Begin SubSubSection
\begin{enumerate}[{MS}1. ]
	\item 
\end{enumerate}
% End SubSubSection

% End SubSection

\subsection{Security Requirements}
\label{sub:security_requirements}
% Begin SubSection

\subsubsection{Access Requirements}
\label{ssub:access_requirements}
% Begin SubSubSection
\begin{enumerate}[{SR}1. ]
	\item 
\end{enumerate}
% End SubSubSection

\subsubsection{Integrity Requirements}
\label{ssub:integrity_requirements}
% Begin SubSubSection
\begin{enumerate}[{SR}1. ]
	\item 
\end{enumerate}
% End SubSubSection

\subsubsection{Privacy Requirements}
\label{ssub:privacy_requirements}
% Begin SubSubSection
\begin{enumerate}[{SR}1. ]
	\item 
\end{enumerate}
% End SubSubSection

\subsubsection{Audit Requirements}
\label{ssub:audit_requirements}
% Begin SubSubSection
\begin{enumerate}[{SR}1. ]
	\item 
\end{enumerate}
% End SubSubSection

\subsubsection{Immunity Requirements}
\label{ssub:immunity_requirements}
% Begin SubSubSection
\begin{enumerate}[{SR}1. ]
	\item 
\end{enumerate}
% End SubSubSection

% End SubSection

\subsection{Cultural and Political Requirements}
\label{sub:cultural_and_political_requirements}
% Begin SubSection

\subsubsection{Cultural Requirements}
\label{ssub:cultural_requirements}
% Begin SubSubSection
\begin{enumerate}[{CP}1. ]
	\item 
\end{enumerate}
% End SubSubSection

\subsubsection{Political Requirements}
\label{ssub:political_requirements}
% Begin SubSubSection
\begin{enumerate}[{CP}1. ]
	\item 
\end{enumerate}
% End SubSubSection

% End SubSection

\subsection{Legal Requirements}
\label{sub:legal_requirements}
% Begin SubSection

\subsubsection{Compliance Requirements}
\label{ssub:compliance_requirements}
% Begin SubSubSection
\begin{enumerate}[{LR}1. ]
	\item 
\end{enumerate}
% End SubSubSection

\subsubsection{Standards Requirements}
\label{ssub:standards_requirements}
% Begin SubSubSection
\begin{enumerate}[{LR}1. ]
	\item 
\end{enumerate}
% End SubSubSection

% End SubSection

% End Section

\appendix
\section{Division of Labour}
\label{sec:division_of_labour}
% Begin Section
Include a Division of Labour sheet which indicates the contributions of each team member. This sheet must be signed by all team members.
% End Section

\newpage
\section*{IMPORTANT NOTES}
\begin{itemize}
	\item Be sure to include all sections of the template in your document regardless whether you have something to write for each or not
	\begin{itemize}
		\item If you do not have anything to write in a section, indicate this by the \emph{N/A}, \emph{void}, \emph{none}, etc.
	\end{itemize}
	\item Uniquely number each of your requirements for easy identification and cross-referencing
	\item Highlight terms that are defined in Section~1.3 (\textbf{Definitions, Acronyms, and Abbreviations}) with \textbf{bold}, \emph{italic} or \underline{underline}
	\item For Deliverable 1, please highlight, in some fashion, all (you may have more than one) creative and innovative features. Your creative and innovative features will generally be described in Section~2.2 (\textbf{Product Functions}), but it will depend on the type of creative or innovative features you are including.
\end{itemize}


\end{document}
%------------------------------------------------------------------------------