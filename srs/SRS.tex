\documentclass[]{article}

% Imported Packages
%------------------------------------------------------------------------------
\usepackage{amssymb}
\usepackage{amstext}
\usepackage{amsthm}
\usepackage{amsmath}
\usepackage{enumerate}
\usepackage{fancyhdr}
\usepackage[margin=1in]{geometry}
\usepackage{graphicx}
\usepackage{extarrows}
\usepackage{setspace}
%------------------------------------------------------------------------------

% Header and Footer
%------------------------------------------------------------------------------
\pagestyle{plain}  
\renewcommand\headrulewidth{0.4pt}                                      
\renewcommand\footrulewidth{0.4pt}                                    
%------------------------------------------------------------------------------

% Title Details
%------------------------------------------------------------------------------
\title{T02 Group 5 Deliverable \#1}
\author{SE 3A04: Software Design II -- Large System Design}
%------------------------------------------------------------------------------

% Document
%------------------------------------------------------------------------------
\begin{document}

\maketitle	

\section{Introduction}
\label{sec:introduction}
% Begin Section

The following is a description of the product to be developed, as well as an overview of the SRS.

\subsection{Purpose}
\label{sub:purpose}
	SpaceBase Ephemeris (SBE) is a mobile simulation game that models the overall behavior of a space settlement on a celestial body, i.e. a real-life system. The game represents how different sub-systems interact with each other and affect the overall system. It replicates how each of the sub-systems of a space base react to different stimuli from outside of the system. Therefore, the player must ensure that the sub-systems are working in the desired fashion to keep the base operating. The purpose of this document is to provide a general description of this project and to specify the requirements for this game. It is meant to be a form of communication between the developers of this game and its clients, i.e. Dr. Ridha Khedri and the teaching assistants for SFWR ENG 3A04. The SRS is also meant to be used as a reference by the developers to ensure that the specified requirements have been fulfilled.

\subsection{Scope}
\label{sub:scope}
	The focus of this SRS is the development of the software product, SBE. The game mimics the operations of a space settlement, and places the user as the head of the base. Hence, the key duty of the player is to ensure that the base remains operating. The player does this by keeping an eye on all the sub-systems and maintaining them whenever there is a need. This will be done by assigning tasks to all the members of the base, which could be anything from fixing a breach in the walls of the base to interaction with alien flora and fauna. If there are multiple stimuli happening concurrently, then the player must prioritize the tasks, based on how critical they are and on what level they affect the safety of the community. The main objective of this game is to provide entertainment to its users. However, it will also develop time management skills of the users, as well as improve their multitasking abilities. 

\subsection{Definitions, Acronyms, and Abbreviations}
\label{sub:definitions_acronyms_and_abbreviations}
\begin{enumerate}
	\item SBE: SpaceBase Ephemeris, the title of the game
\end{enumerate}

\subsection{References}
\label{sub:references}
	No references were used for this document.

\subsection{Overview}
\label{sub:overview}
	This document describes the product software that is to be developed as well as the requirements specifications.
\begin{itemize}
	\item The second section of this document provides an overall description of the game. It provides product perspective when compared to another related product. It describes major functions of the product as well as the user characteristics. It also explains the constraints of the system as well as the assumptions for this project.
	\item The third section of this document specifies the functional software requirements of this game. It provides sufficient details to design a system with the specific requirements and to test that the design has fulfilled those requirements. 
	\item The final section of the document is the non-functional requirements, which vary from different perceptions and are used to ensure successful integration of the game into society.
	\item A Division of Labour section is also placed at the end of the document which details the contributions of each team member.
\end{itemize}

% End Section

\section{Overall Description}
\label{sec:overall_description}

The following is a general description of the product and its requirements. For more specific requirements, see the requirements sections.

\subsection{Product Perspective}
\label{sub:product_perspective}
	SBE is an independent and totally self-contained system. It does not require network communication to function normally. The game will contain elements of various existing simulation games such as [[dwarf fortress]], but is not intended to completely emulate any of them. The simulation is intended for entertainment purposes only, so bears only a thematic relationship to some scientific simulation software. As the game is an Android app, it may be distributed on the Google Play Store, but that does not constitute part of the system.

\subsection{Product Functions}
\label{sub:product_functions}
	\begin{enumerate}
		\item The user will be able to view compositions of various sub-views of the system. Each sub-view corresponds to a different subsystem.
		\item The user will be able to stimulate the system. Each stimuli will be able to one or many subsystems, with reactions cascading appropriately. Each subsystem will have at least 1 stimuli.
		\item The application will simulate the various subsystems and their interactions.
		\item Major ways the user will be able to interact with the system include: expanding their station, issuing orders to the population, and managing power and atmospheric controls of their station.
		\item \textbf{Creative/Innovative:} The user will be able to save the state of their game, and share the corresponding file with other users, who will in turn be able to load it.
	\end{enumerate}

\subsection{User Characteristics}
\label{sub:user_characteristics}
	Users are expected to have at least a high school level diploma and reading level. Users are expected to be generally familiar with the Android operating system and Android apps. Users are expected to have only a cursory understanding of the subject matter, as SBE is intended for entertainment rather than scientific simulation.

\subsection{Constraints}
\label{sub:constraints}
	The following are constraints on the development of the system.
	\begin{enumerate}
		\item The system must be produced as an Android app.
		\item The system must consist of several separate subsystems. (at least 3)
	\end{enumerate}

\subsection{Assumptions and Dependencies}
\label{sub:assumptions_and_dependencies}
	The following are assumptions that affect the requirements for the system.
	\begin{enumerate}
	\item It is assumed that the device running the application will have the Android operating system available.
	\item It is assumed that the application will be run with sufficient privileges to read and write necessary files on the device.
	\item  It is assumed that the device will have access to the Google Play Store (or an alternate distribution method if one is chosen).
	\end{enumerate}

\subsection{Apportioning of Requirements}
\label{sub:apportioning_of_requirements}
	The following requirements may be delayed until future versions of the system.
	\begin{enumerate}
		\item Functionality allowing the user to showcase their system (for example on social media).
		\item Functionality allowing the user to swap between multiple saved states within the application.
		\item Non-token graphical features.
		\item More than the minimum number of subsystems.
		\item Ability to enable or disable subsystems at runtime.
	\end{enumerate}

\section{Functional Requirements}
\label{sec:functional_requirements}
% Begin Section

\begin{enumerate}
	\item The user launches the application
	\begin{enumerate}
		\item User
			\begin{enumerate}
				\item The system provides a view of all the sub--systems in their present state to the user
				\item The system should allow the user to select a sub-system to interact with
			\end{enumerate}
	\end{enumerate}
	\item The user wants to interact with a sub-system
	\begin{enumerate}
		\item User
			\begin{enumerate}
				\item The system should show all stimuli that can stimulate the sub--system
				\item The system should allow the user to select a stimuli
				\item The system must allow the user to control the stimuli
				\item The system must respond to the stimuli
			\end{enumerate}
	\end{enumerate}
	\item The user wants to update the application to the latest version
	\begin{enumerate}
		\item User
			\begin{enumerate}
				\item The system must prompt the user to update the system when the application is launched
			\end{enumerate}
	\end{enumerate}
	\item The user wants to change settings of the system
	\begin{enumerate}
		\item User
			\begin{enumerate}
				\item The system should provide an interface for editing values of how the system works based on customizable aspects of the system
				\item The system should store these new settings and incorporate them into the system’s functionality
			\end{enumerate}
	\end{enumerate}
	\item The user wants to share snapshot of the system
	\begin{enumerate}
		\item Social Media Application
			\begin{enumerate}
				\item The system should be able to provide data that is compatible with the specified social media framework
			\end{enumerate}
			\item User
			\begin{enumerate}
				\item The system should provide a means of sharing user’s system data via social media
			\end{enumerate}
	\end{enumerate}
	\item The user wants to shut down the system
	\begin{enumerate}
		\item User
			\begin{enumerate}
				\item The system should provide a way to close the current running of the system’s processes
				\item The system should provide a way to save progress within the system and sub--systems
			\end{enumerate}
	\end{enumerate}
\end{enumerate}

\section{Non-Functional Requirements}
\subsection{Look and Feel Requirements}
\subsubsection{Appearance Requirements}
\begin{enumerate}
\item LF1. The application must have graphical interface (not just textual) for the users to use.
\item LF2. The graphical interface must be spannable, zoomable, and rotatable.
\item LF3. The interface must be isometric or top-down view.
\item LF4. The product shall be attractive to all audiences. 
\end{enumerate}

\subsubsection{Style Requirements}
\begin{enumerate}
	\item S1. N.A.
\end{enumerate}

\subsection{Usability and Humanity Requirements}
\subsubsection{Ease of Use Requirements}
\begin{enumerate}
	\item UH1. The application shall be easy for anybody of age 13 and up to use.
	\item UH2. The application shall include a brief tutorial or manual to guide new users.
	\item UH3. The application shall introduce game mechanics step-by-step so the users are not overwhelmed.
	\item UH4. The application shall be used by anyone, by substituting symbols with language. 
\end{enumerate}

\subsubsection{Personalization and Internalization Requirements}
\begin{enumerate}
 \item UH1. The application shall create memory space in the device to store data.
 \item UH2. There will only be English option, however everything shall be understood through symbolic icons.
 \item UH3. Personal settings or game files shall be able to be saved, and loaded at user�s will.
\end{enumerate}

\subsubsection{Learning Requirements}
\begin{enumerate}
	\item LER1. This application shall be easy for anyone over 13 to learn.
	\item LER2. The application will provide a short learning tutorial to ease of any confusion.
\end{enumerate}

\subsubsection{Understandability and Politeness Requirements}
\begin{enumerate}
	\item UPR1. The application must use symbols and words that are naturally understood.
	\item UPR2. Any abbreviations or acronyms must be easily looked up or understood.
\end{enumerate}

\subsubsection{Accessibility Requirements}
\begin{enumerate}
	\item ACCR1. This application shall conform to the Canadians with Disabilities Act.
\end{enumerate}

\subsection{Performance Requirements}
\subsubsection{Speed and Latency Requirements}
\begin{enumerate}
	\item SOLR1. Any interaction between the user and the system must have a maximum response time of 1 second.
	\item SOLR2. Any interaction relating to changing the view of the system must have an instant response time.
\end{enumerate}

\subsubsection{Safety-Critical Requirements}
\begin{enumerate}
	\item SCR1. Any colour scheme must not induce dizziness, tiredness, or seizures.
	\item SCR2. The graphics must be low enough to not overheat the GPU.
\end{enumerate}

\subsubsection{Precision or Accuracy Requirements}
\begin{enumerate}
	\item PAR1. All monetary values must be to two decimal places.
	\item PAR2. Any other critical values the system depends on must be to six significant digits.
\end{enumerate}

\subsubsection{Reliability and Availability Requirements}
\begin{enumerate}
	\item PR1. The application must not crash before saving user progress and data.
	\item PR2. No data must be lost even in case of system failure.
	\item PR3. This application must be available 24 hours per day, 365 days per year. 
\end{enumerate}

\subsubsection{Robust or Fault-Tolerance Requirements}
\begin{enumerate}
	\item RFTR1. N.A.
\end{enumerate}

\subsubsection{Capacity Requirements}
\begin{enumerate}
	\item CAPR1. The application shall be at maximum 300MB.
\end{enumerate}

\subsubsection{Scalability or Extensibility Requirements}
\begin{enumerate}
	\item SER1. N.A.
\end{enumerate}

\subsubsection{Longevity Requirements}
\begin{enumerate}
	\item LR1. The application is expected to operate as long as the hardware supports it.
\end{enumerate}

\subsection{Operational and Environmental Requirements}
\subsubsection{Expected Physical Environment}
\begin{enumerate}
	\item EP1. N.A.
\end{enumerate}

\subsubsection{Requirements for Interfacing with Adjacent Systems}
\begin{enumerate}
	\item RIAS1. This application must work on at least the last four generations of smartphones.
\end{enumerate}

\subsubsection{Productization Requirements}
\begin{enumerate}
	\item PRR1. This application shall be distributed as APK file.
	\item PRR2. The application must be available on Android Play Store.
\end{enumerate}

\subsubsection{Release Requirements}
\begin{enumerate}
	\item RR1. The application must fix any bugs that are brought forward by its customers.
	\item RR2. There must be a new release every other month, at minimum.
\end{enumerate}

\subsection{Maintainability and Support Requirements}
\subsubsection{Maintenance Requirements}
\begin{enumerate}
	\item MSM1. A new Module Interface Specification must be made within a week of the date when the requirements are agreed upon.
	\item MSM2. Any dependent system must be able to be identified through all of its independent systems in order to solve any bugs.
\end{enumerate}

\subsubsection{Supportability Requirements}
\begin{enumerate}
	\item MSS1. An email system where users can interact with the developing team must be set up.
\end{enumerate}

\subsubsection{Adaptability Requirements}
\begin{enumerate}
	\item ADR1. This application is to run under Android.
	\item ADR2. The application may eventually be ported to Windows, Mac, Linux, or iOS.
\end{enumerate}

\subsection{Security Requirements}
\subsubsection{Access Requirements}
\begin{enumerate}
	\item SAR1. N.A.
\end{enumerate}

\subsubsection{Integrity Requirements}
\begin{enumerate}
	\item IR1. The application shall ensure that any attempts to breach any personal data be deterred.
	\item IR2. The application will make sure that any stored data is correct to desired system.
\end{enumerate}

\subsubsection{Privacy Requirements}
\begin{enumerate}
	\item PR1. The application shall not collect any personal information.
	\item PR2. The application will notify the user if any agreement policy changes.
\end{enumerate}

\subsubsection{Audit Requirements}
\begin{enumerate}
	\item AR1. N.A.
\end{enumerate}

\subsubsection{Immunity Requirements}
\begin{enumerate}
	\item IMR1. N.A.
\end{enumerate}

\subsection{Cultural and Political Requirements}
\subsubsection{Cultural Requirements}
\begin{enumerate}
	\item CR1. This application shall not contain any symbols or messages that may offend any ethnic or religious groups.
	\item CR2. The application must properly distinguish characters from different origins.
\end{enumerate}

\subsubsection{Political Requirements}
\begin{enumerate}
	\item PP3. N.A.
\end{enumerate}

\subsection{Legal Requirements}
\subsubsection{Compliance Requirements}
\begin{enumerate}
	\item LCR1. N.A.
\end{enumerate}

\subsubsection{Standards Requirements}
\begin{enumerate}
	\item LSR1. N.A.
\end{enumerate}

\appendix
\section{Division of Labour}
\label{sec:division_of_labour}
\begin{enumerate}
	\item Arfa: Introduction
	\item Ian: Overall Description
	\item Areeb: Functional Requirements
	\item Nishanth: Functional Requirements
	\item Steven: Nonfunctional Requirements
\end{enumerate}

\end{document}
%------------------------------------------------------------------------------
